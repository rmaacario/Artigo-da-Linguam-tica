\documentclass[a4paper, twocolumn, 11pt, twoside]{article}

% update your language here.
\usepackage[brazil]{babel}
\usepackage{linguamatica}

% This assumes your files are encoded as UTF8
\usepackage[utf8]{inputenc}
\usepackage{amssymb} % For symbols like ? in glosses

\usepackage{float}

\usepackage[T1]{fontenc}
\usepackage{xcolor}   % For color
\usepackage{graphicx} % For handling images (optional)
\usepackage{fancyhdr} % For custom headers/footers (optional)
%\usepackage{natbib}
\usepackage{linguex}
\usepackage{hyperref}

\definecolor{ForestGreen}{RGB}{34,139,34}
\definecolor{Maroon}{RGB}{128,0,0}

\definecolor{lightgray}{RGB}{224, 224, 224}
\definecolor{mblue}{RGB}{0, 127, 204}
\definecolor{lightyellow}{RGB}{255, 247, 209}
\definecolor{lightblue}{RGB}{204, 230, 255} % 

%% Please use UTF8 in your document

%% Set the bibliography style.
%% Styles provided for Spanish (Castilian), Catalan and Portuguese.
%% Other languages will be created on a required basis.
%\usepackage[backend=biber]{biblatex}
%\addbibresource{references.bib}

\bibliographystyle{sp_por}   % This for Portuguese

%% You can leave this unchanged. It will be updated by the editors when your paper gets published.
\submitted{15 de \OCT{} 2017}
\accepted{3 de \DEC{} 2017}

%% Add your title in the main language used in the article
\title{Desafios na Tradução da Informação Espacial do EN para o PT-br}

%% Currenty the title in English is also mandatory
\titleEN{Challenges in Translating Spatial Information from EN to PT-br}


%% Add authors here. Three lines for each author.
%% First line with author name, second with the affiliation and third with e-mail
%% In cases where a second line is needed for the affiliation, use the command \nl to separate them. 
\author{
  Rafael Fernandes
  \instituto{Universidade de São Paulo}
  \email{rafael.macario@usp.br} 
  \and 
  Rodrigo Souza
 \instituto{Universidade de São Paulo}
  \email{rodrigo.aparecido.souza@usp.br}
  \and
  Marcos Lopes
  \instituto{Universidade de São Paulo}
  \email{marcoslopes@usp.br}
  \and
%%  Paulo Santos
%%  \instituto{Flinders University}
%%  \email{paulo.santos@flinder.edu.au}
%%%%  \and
%%  Thomas Finbow
%%  \instituto{Universidade de São Paulo}
%%  \email{thomas.finbow@usp.br}
}


\begin{document}


% Indentation for package "linguex"
\setlength{\Exlabelwidth}{0.2em}
\setlength{\SubExleftmargin}{1.5em}

\maketitle

%% Add the abstract in the main language for the article. If possible, doesn't add any citations here.
\begin{resumo}
  A Tradução Automática Neural (TAN), apesar de ser a abordagem dominante, ainda enfrenta desafios significativos ao traduzir conhecimento espacial. Neste estudo, utilizamos o Raciocínio Espacial Qualitativo (REQ) para representar e analisar informações espaciais em traduções automáticas do inglês para o português. Foram traduzidas 145 frases dos corpora CAM e COCA, utilizando os sistemas Google Translate e DeepL. Ao mapear logicamente as diferenças de significado entre as traduções e os originais, utilizando o REQ, identificamos que a TAN comete em média 10,6\% de erros semânticos e 12\% de erros de projeção sintática em traduções envolvendo conceitos espaciais. Nossos resultados evidenciam a necessidade de aprimorar os modelos de TAN para lidar com as nuances da linguagem espacial, contribuindo para o avanço da pesquisa em tradução automática.
\end{resumo}

%% Add keywords in the article main language, in lowercase
\palavraschave{tradução automática neural; semântica espacial; raciocínio espacial qualitativo; tradução automática inglês-português; polissemia; tipologia linguística}

%% Add the abstract in English
\begin{abstract}
  Neural Machine Translation (NMT), currently the leading approach, still faces challenges in translating spatial knowledge. In this study, we used Qualitative Spatial Reasoning (QSR) to represent spatial information in automatic translations from English to Portuguese. We translated $145$ sentences from the CAM and COCA corpora using Google Translate and DeepL, and identified the causes of unnatural translations. Using QSR, we logically mapped the differences in meaning. Our results indicate that, despite generally good performance, NMT struggles with specific spatial meanings, resulting in $10.6\%$ semantic errors and $12.0\%$ syntactic projection errors. This work explores the practical and theoretical challenges of machine translation.
\end{abstract}

%% add the keywords in English
\keywords{neural machine translation; spatial semantics; qualitative spatial reasoning; english-portuguese machine translation; polysemy; language typology}


\section{Introdução}

A Tradução Automática Neural (TAN) tornou-se o paradigma dominante na área de Tradução Automática, tanto em estudos acadêmicos quanto em aplicações práticas \citep{dabre2020survey}. Esse avanço se deve, principalmente, à capacidade aprimorada dos modelos de aprendizado profundo em capturar dependências de longo alcance nas frases \citep{vaswani2017attention, yang2020survey}.

No entanto, apesar dos avanços, alguns tradutores automáticos ainda enfrentam desafios ao lidar com as nuances da linguagem espacial, como a polissemia das preposições e a projeção idiossincrática da maneira de movimento em inglês diretamente para verbos em português \citep{McCleary-Viotti-2004}. Um exemplo disso pode ser visto no Exemplo (1), retirado do Cambridge Online Dictionary (CAM), onde a tradução do inglês (EN) para o português (PT) foi realizada com o Google Translate (GT) e o DeepL (DL).

\footnotesize
  \ex. He swam \emph{across} the river. (CAM)\label{ex:across2}
  \ag. ?~Ele nadou do outro lado do rio. (GT) \\  
  3SG.M swam from-the other side of-the river\\ 
  \bg. Ele atravessou o rio a nado. (DL) \\ 
  3SG.M crossed the river by swimming\\
  \par
\normalsize
  
A tradução do Exemplo~\ref{ex:across2} feita pelo modelo GT, embora gramaticalmente correta, erra ao não capturar a expressão mais natural em PT para a sentença em EN. O DL, por outro lado, acerta.
  
A razão por trás dessa tradução errada está na polissemia da preposição ACROSS, que pode significar tanto uma localização oposta, fixa ao ponto de referência quanto movimento de um lado de um espaço para o outro. Neste caso em particular, o significado pretendido é claramente o último. Para ilustrar isso, vamos considerar as Figuras 1 e 2.

\begin{figure}[ht]
  \centering
  \includegraphics[width=0.3\textwidth]{aa-a-2.jpg}
  \caption{Diagrama semântico de (1)-a.}\label{fig:across-1a}
\end{figure}

\begin{figure}[ht]
  \centering
  \includegraphics[width=0.3\textwidth]{bb-b-2.jpg}
  \caption{Diagrama semântico de (1)-b.}\label{fig:across-1b}
\end{figure}


A Figura~\ref{fig:across-1a} representa o GT, que indica movimento dentro de um local específico (uma margem oposta do rio). No entanto, a Figura~\ref{fig:across-1b}, representando a saída DL, transmite o significado de cruzar de uma margem do rio para a outra, capturando assim a natureza dinâmica implícita na frase original EN. 

Com isso em mente, este artigo explora a tradução automática de frases em EN que envolvem informações espaciais (topologia ou movimento) para PT, utilizando GT e DL. Nosso objetivo é duplo: primeiro, baseados nos trabalhos de Spranger et al. (2016), Freksa e Kreutzmann (2016) e Randell et al. (1992), formalizamos amostras de frases nas línguas de origem e destino. Em seguida, categorizamos as traduções para identificar erros comuns cometidos por ferramentas de TAN. Em vez de focar no processo de TAN em si, discutiremos os significados espaciais que essas ferramentas têm dificuldade em capturar, iluminando práticas e direções teóricas para pesquisa em linguagem espacial e TA. Nossos resultados mostram que, apesar do bom desempenho geral, os motores de TA ainda cometem erros sistemáticos em algumas categorias ao traduzir textos de EN para PT.


\subsection{Desafios na Tradução da Espacialidade}

\section{Metodologia}
Nesta seção, apresentamos as etapas metodológicas do nosso trabalho, composta pela coleta dos dados, pela classificação das preposições, pelo processo de tradução, pelas formalizações das informações espaciais e pela categorização das traduções. 

\subsection{Coleta dos Dados}

Para compor nosso corpus, compilamos $145$ sentenças contendo cinco preposições em EN que transmitem conhecimento espacial: 
\textit{across}, \textit{into}, \textit{to}, \textit{through} e \textit{via}. As sentenças foram obtidas do Cambridge Online Dictionary (CAM)\footnote{\url{https://dictionary.cambridge.org/}} e 
do Corpus of Contemporary American English (COCA)\footnote{\url{https://www.english-corpora.org/coca/}}. 
Manualmente, anotamos cada sentença de acordo com o significado ligado às preposições, significado espacial e também atribuímos um identificador para o número da sentença.
No Exemplo~\ref{example:through1}, apresentamos uma das sentenças do corpus (amostra \texttt{Through-CAM-1-2}).

\footnotesize
\ex. He struggled through the crowd till he reached the front. (CAM) \label{example:through1} 
\ag. ?~Ele lutou no meio da multidão até chegar à frente. (GT) \\ 
3SG.M fought in-the middle of-the crowd until reach to-the front \\ 
\medskip
\bg. ?~Ele se debateu entre a multidão até chegar à frente. (DL) \\ 
3SG.M REFL struggled amongst the crowd until reach to-the front \\ 
\par
\normalsize

\subsection{Classificação das Sentenças}


Sistematicamente, categorizamos cada sentença com base nos significados espaciais alinhados às entradas encontradas no CAM para cada preposição, 
como mostra a Tabela~\ref{table:prep-categorizations}.

\begin{table}[H]
  \footnotesize
  \centering
  \begin{tabular}{|l|p{.25\textwidth}|}
  \hline
  \textbf{Preposição (EN)} & \textbf{Significado(s) Espaciais(s)}
  \\\hline\hline
  Across  & \begin{tabular}[c]{@{}l@{}}(1) posição perpendicular \\ (2) movimento de atravessar\\ (3) localização oposta \\ (4) em todas as partes de\end{tabular}\\\hline
  
  Into & \begin{tabular}[c]{@{}l@{}}(1) movimento para um \\ ponto não especificado de \\ uma área ou contêiner \\ (2) movimento em direção a uma \\ barreira ou obstáculo\end{tabular}\\\hline
  
  Onto & \begin{tabular}[c]{@{}l@{}}(1) movimento sobre uma \\superfície \\  sair de uma área \\ delimitada\end{tabular} \\\hline
  
  Through & \begin{tabular}[c]{@{}l@{}}(1) movimento de atravessar \\ uma área de uma \\extremidade a outra  \\ (2) movimento de passar ou \\ penetrar uma barreira
  \end{tabular} \\\hline
  
  Via & (1) parte de uma rota\\\hline                \end{tabular}
  \caption{Categorização das preposições \textit{across}, \textit{into}, \textit{onto}, \textit{through} e \textit{via} baseada nas definições do CAM.}
  \label{table:prep-categorizations}
  \end{table}

\subsection{Tradução das Sentenças}

As sentenças foram traduzidas com o Google Translate (GT)\footnote{\url{https://translate.google.com}} e 
com o DeepL (DL)\footnote{\url{https://www.deepl.com/translator}}, utilizando suas versões publicamente 
disponíveis on-line em agosto-setembro de 2023. Além disso, para facilitar a comparação entre as traduções, fornecemos 
referências profissionais traduzidas por humanos para todas as sentenças.

%\subsection{Formalização das Informações Espaciais}

\subsection{Formalização das Sentenças}

A partir do trabalho de~\citet{spranger2016robust}, definimos cada intervalo de tempo $t$ como um conjunto de pontos e utilizamos o predicado \textit{occurs\_in($\theta$, $t$)} para denotar que um evento $\theta$ ocorre durante um intervalo de tempo $t$.
Com base em~\citet{freksa2016neighborhood}, definimos os eventos $\theta$ por meio do cojunto de treze relações qualitativas espaço-temporais apresentadas na Figura~\ref{fig:13_relations}.

\begin{figure}[ht]
	\centering
	\includegraphics[width=\columnwidth]{Figures/13_concep_neigb_relat.png}
	\caption{Treze relações qualitativas entre dois objectos lineares estendidos sobre uma reta orientada~\citep{freksa2016neighborhood}.}
	\label{fig:13_relations}
\end{figure}

A Figura~\ref{fig:13_relations} apresenta  as treze relações conjuntamente exaustivas e em pares disjuntivos baseadas no Cálculo de Intervalos de Allen~\citep{allen1983maintaining}.
Essas relações podem ser descritas pelo seguinte cojunto: \{\textit{before, after, equal, meets, met by, overlaps, overlapped by, during, contains, starts, started by, finishes, finished by}\}. 
Com esse cojunto de relações, podemos representar transições relacionadas ao movimento de objetos que fazem parte de um evento.

Por \textit{default}, assumimos um espaço 3D para todos os objetos em nossas representações de movimento nas cenas. Para representar as informações espaciais em sentenças como as do Exemplo (1), em que a preposição ``across'' denota
o movimento de atravessar uma superfície, definimos uma função \textit{surface(r)}. Essa função mapeia relações como
\textit{during} ou \textit{contains}, projetando um objeto em uma superfície 2D.

Para modelar relações mereotopológicas, nos baseamos no RCC-8~\citep{randell1992spatial}: \{\textit{dc, ec, po, eq, tpp, ntpp, tpp}${^{-1}}$, \textit{ntpp}${^{-1}}$\}.
Para formalizar uma sentença como a gerada pela tradução do GT no Example~\ref{example:across2}, nós definimos uma Região de Referência (\textit{RR}), que é uma parte de 
uma região \textit{R}, ou Fundo, localizada fora da região onde a ação executada pelo objeto \textit{F}, a Figura, 
acontece. A Região de Referência é separada do restante de \textit{R} por uma linha transversal (chamada por nós de \textit{meridiano}) 
que liga com \textit{R} em dois pontos distantes (não consecutivos) e não toca $F: R_{op} = ntpp(F, R)$.

De modo a representar a relação entre o predicado \textit{occurs\_in($\theta$, $t$)} e as relações qualitativas 
apresentadas na Figura~\ref{fig:13_relations}, nós utilizamos o conectivo $\leftlsquigarrow$, que denota uma implicação
revogável, isto é, uma forma de raciocíno que é racionalmente convincente, mas carece de validade dedutiva. 
Nesse contexto, as premissas do argumento oferecem suporte racional para a conclusão, mas há a possibilidade 
de as premissas serem verdadeiras e a ser falsa. Em resumo, a conexão entre as premissas e a conclusão são provisórias
e podem ser anuladas por informações suplementares.  

\subsection{Categorização das Traduções}
%\subsection{title}

Categorizamos as $145$ sentenças traduzidas pelo GT e pelo DL (ou seja, $290$ no total), comparando as traduções das preposições  
com seus respectivos significados apresentados na Tabela~\ref{table:prep-categorizations}. Para tanto, utilizamos 
as seguintes categorias: tradução (C)orreta, tradução equivocada de (S)ignificado e erro de (P)rojeção na tradução. 
A última envolve principalmente a incorporação inadequada da maneira do movimento no verbo das sentenças traduzidas para o português, em vez de representar essa informação com adjuntos (veja, por exemplo,~\ref{table:FormThrough}).

\section{Resultados e Discussão}
Inicialmente, para sintetizar os resultados obtidos por meio da nossa análise formal, discutiremos as
formalizações das sentenças nos Examplos~\ref{example:across2} and \ref{example:through1}.  As fórmulas apresentadas a seguir descrevem relações espaciais qualitativas 
entre as sentenças originais e suas respectivas traduções automáticas.
Em ambas as formalizações, os intervalos de tempo foram representados por $t_1$, $t_2$, $t_3$, onde $t_1$ e $t_3$ correspondem aos intervalos inicial e final, respectivamente. 
Já $t_2$, por sua vez,  representa um intervalo de tempo entre $t_1$ e $t_3$.
A tabela~\ref{table:FormAcross} mostra as fórmulas que representam o Exemplo~\ref{example:across2}.

\begin{table}[!htb]
  \centering
  \scalebox{0.8}{
  \begin{tabular}{| l |p{.3\textwidth}|}%{|l|l|}
  \hline
  \textbf{Original text:} He swam \underline{across} the river. 
  \\\hline\hline
  \begin{tabular}[c]{@{}l@{}}  
  $\forall t$ $ \in \{t_1, t_2, t_3\}, t_1 < t_2 < t_3$\\ 
  $occurs\_in(moves\_across(he, river), t)$ $\leftlsquigarrow$\\
  $river' = surface(river)$ $\wedge$\\
  $starts(he, river', t_1)$ $\wedge$\\ 
  $during(he, river', t_2)$ $\wedge$\\ 
  $finishes(he, river', t_3)$\\
  %$continuous(t_1, t_2, t_3)$\\
  
  \end{tabular}\\\hline\hline
  \textbf{GT:} Ele nadou \underline{do outro lado} do rio. 
  \\\hline\hline
  \begin{tabular}[c]{@{}l@{}}  
  
  $\forall t$ $\in \{t_1, t_2, t_3\}, t_1 < t_2 < t_3$\\
  $occurs\_in(moves\_on\_opposite\_side(he, river_{op}), t)$ $\leftlsquigarrow$\\
  $river' = surface(river_{op})$ $\wedge$\\
  $starts(he, river', t_1)$ $\wedge$\\
  $during(he, river', t_2)$ $\wedge$\\ 
  $finishes(he, river', t_3)$\\
  \end{tabular}\\\hline\hline
  
  \textbf{DL:} Ele \underline{atravessou} o rio a nado.
  \\\hline\hline
  \begin{tabular}[c]{@{}l@{}} 
  $\forall t$ $ \in \{t_1, t_2, t_3\}, t_1 < t_2 < t_3$\\ 
  $occurs\_in(moves\_across(he, river_{op}), t)$ $\leftlsquigarrow$\\
  $river' = surface(river_{op})$ $\wedge$\\
  $starts(he, river', t_1)$ $\wedge$\\ 
  $during(he, river', t_2)$ $\wedge$\\ 
  $finishes(he, river', t_3)$\\
  % $continuous(t_1, t_2, t_3)$\\
  \end{tabular}\\\hline
  
  \end{tabular}}
  \caption{Formalizações das sentenças do Exemplo~\ref{example:across2}.}
  \label{table:FormAcross}
  \end{table}
 
A formalização na Tabela~\ref{table:FormAcross} nos permite representar a diferença lexical mencionada na Seção~\ref{sec:introduction}. 
Na sentença original, a preposição \textit{across} é categorizada como sentido~(2) de acordo com CAM (Tabela~\ref{table:prep-categorizations}). 
Entretanto, a tradução da GT opta pelo sentido~(3). A expressão ``do outro lado'' transmite o significado de 
que a ação executada pelo indivíduo, \textit{ele}, ocorreu em uma porção do \textit{rio} que é separada da \textit{RR}, 
diferindo da região onde a ação ocorreu na sentença original, e alinhando-se com a tradução da DL.

Na formalização das sentenças do Exemplo~\ref{example:through1}, apresentada na Tabela~\ref{table:FormThrough},
também é possível perceber diferenças qualitativas relacionadas a informação espacial. No exemplo, \textit{through} é empregado no sentido~(1) (da Tabela~\ref{table:prep-categorizations}).

% Through
\begin{table}[!htb]
  \centering
  \scalebox{0.7}{
  \begin{tabular}{| l |p{.3\textwidth}|}%{|l|l|}
  \hline
  \textbf{Original text:} He struggled \underline{through} the crowd till he\\ reached the front.
  \\\hline\hline
  \begin{tabular}[c]{@{}l@{}}  
  $\forall t$ $\in \{t_1, t_2, t_3\}, t_1 < t_2 < t_3$\\ $occurs\_in(arduously(moves\_through(he, crowd), t))$ $\leftlsquigarrow$\\ 
  $starts(he, crowd, t_1)$ $\wedge$\\  
  $during(he, crowd, t_2)$ $\wedge$\\ 
  $finishes(he, crowd, t_3)$ \\
  \end{tabular}\\\hline\hline
  
  \textbf{GT:} Ele lutou \underline{no meio} da multidão até chegar à frente. 
  \\\hline\hline
  \begin{tabular}[c]{@{}l@{}} 
  $\forall t$ $\in \{t_1, t_2, t_3\}, t_1 < t_2 < t_3$\\
  $occurs\_in(fights(he, crowd)$ $\wedge$ $moves\_to(he, crowd), t)$ $\leftlsquigarrow$\\
  $starts(he, crowd, t_1)$ $\wedge$\\  
  $during(he, crowd, t_2)$ $\wedge$\\ 
  $finishes(he, crowd, t_3)$\\ 
  \end{tabular}\\\hline\hline
  
  \textbf{DL:} Ele se debateu \underline{entre} a multidão até chegar à frente.
  \\\hline\hline
  \begin{tabular}[c]{@{}l@{}} 
  $\forall t$ $\in \{t_1, t_2, t_3\}, t_1 < t_2 < t_3$\\
  $occurs\_in(flounder(he, crowd)$ $\wedge$ $moves\_to(he, crowd), t)$ $\leftlsquigarrow$\\
  $starts(he, crowd, t_1)$ $\wedge$\\  
  $during(he, crowd, t_2)$ $\wedge$\\ 
  $finishes(he, crowd, t_3)$\\  
  \end{tabular}\\\hline
  \end{tabular}}
  \caption{Formalizações para as sentenças do Exemplo~\ref{example:through1}.}
  \label{table:FormThrough}
  \end{table}

  Como podemos observar, as três relações $\langle starts, during, finishes\rangle$ foram aplicadas
  em todas as sentenças na Tabela~\ref{table:FormThrough}. Essa escolha reflete que a ação começou em um ponto de entrada da \textit{multidão} 
  e terminou em um ponto de saída. As distinções entre as frases do evento estão na maneira como a ação ocorreu.

A sentença original do Exemplo~\ref{example:through1} descreve um evento em que o movimento da figura acontece 
de um ponto dentro ou além da multidão para uma extremidade, em direção à frente da multidão e realizado comdificuldade. 
 Essa dificuldade é incorporada ao verbo \textit{to struggle} do EN. De modo semelhante, os tradutores automáticos, GT e DL, 
optaram por traduzir a mesma informação usando verbos do PT como ``lutar'' (\textit{to fight}) e ``se debater'' (\textit{to flounder}),
respectivamente, produzindo traduções que soam execessivamente exageradas. Uma tradução mais precisa seria 
``Ele atravessou a multidão \emph{com dificuldade} até chegar à frente.'' Nessa versão, o ato de atravessar é expresso por ``atravessar'', 
enquanto a dificuldade da ação é transmitida por ``com dificuldade''.

Para representar essa distinção nas formalizações, introduzimos um predicado de segunda ordem (\textit{arduously}). 
Na Tabela~\ref{table:FormThrough}, o esforço envolvido na execução da ação vinculada ao verbo \textit{lutar} é expresso pelo predicado \textit{arduously}. 
Em contraste, as expressões ``lutou no meio de'' e ``se debateu entre'' denotam eventos individualizados. As informações na tabela sumarizam os desafios que o GT e o DL enfrentaram ao traduzir as informações sobre a maneira dos eventos, 
resultando em erros de tradução. 

A formalização das sentenças do Exemplo, em que ``into'' está presente no texto original, também possibilitou explicitar 
diferenças qualitativas não captadas nas traduções automáticas. Na Tabela~\ref{table:FormInto}, apresentamos a forma lógica de cada 
sentença.

% Into
\begin{table}[!htb]
  \centering
  \scalebox{0.7}{
  \begin{tabular}{| l |p{.3\textwidth}|}%{|l|l|}
  \hline
  \textbf{Original text:} He just about stops himself from flying\\ \underline{into} the wall. 
  \\\hline\hline
  \begin{tabular}[c]{@{}l@{}}  
  $\forall t$ $ \in \{t_1, t_2, t_3\}, t_1 < t_2 < t_3$\\ 
  $occurs\_in(moves\_into(he, wall), t)$ $\leftlsquigarrow$\\
  $before(he, wall, t_1)$ $\wedge$\\ 
  $meets(he, wall, t_2)$ $\wedge$\\ 
  $after(he, wall, t_3)$ 
  \\ 
  \end{tabular}\\\hline\hline
  \textbf{GT:}  Ele quase se impede de voar \underline{para} a parede e\\ amamenta o carro de volta aos boxes.
  \\\hline\hline
  \begin{tabular}[c]{@{}l@{}}  
  
  $\forall t$ $\in \{t_1, t_2\}, t_1 < t_2$\\
  $occurs\_in(moves\_to(he, wall), t)$ $\leftlsquigarrow$\\ %Aqui, para mim, fez mais sentido colocar movest_to
  $before(he, wall, t_1)$ $\wedge$\\ 
  $meets(he, wall, t_2)$ 
  
  \end{tabular}\\\hline\hline
  
  \textbf{DL:} Por pouco ele não voa \underline{contra} o muro e leva o carro\\ de volta para os boxes.
  \\\hline\hline
  \begin{tabular}[c]{@{}l@{}} 
  $\forall t$ $\in \{t_1\}$\\
  $occurs\_in(moves\_to(he, wall), t)$ $\leftlsquigarrow$\\ %Aqui, para mim, fez mais sentido colocar movest_to
  $before(he, wall, t_1)$ $\wedge$\\ 
  $meets(he, wall, t_2)$ $\wedge$\\ 
  $after(he, wall, t_3)$ 
  % $continuous(t_1, t_2, t_3)$\\
  \end{tabular}\\\hline
  
  \end{tabular}}
  
  \caption{Formalizações para as sentenças do Exemplo.}
  \label{table:FormInto}
  \end{table}

As formalizações na Tabela~\ref{table:FormInto} mostram, assim como as dos exemplos anteriores, as dificuldades dos tradutores 
automáticos em traduzir informações espaciais. Na sentença orignal, ``Into'' tem a segunda acepção da Tabela~\ref{table:prep-categorizations}, isto é, 
``movimento em direção a uma barreira ou obstáculo''. Considerando a cena evocada pela sentença, a ideia transmitida é a de que 
esse movimento culminaria em um choque violento. Essa ideia não é captada pela tradução do GT, por exemplo. Nela, a 
preposição ``para'' recupera a informação de que a figura, ``ele'', se movimentou em direção ao fundo, ``a parede''. Porém, essa preposição não indica, necessariamente, um movimento 
em direção a uma barreira. Além disso, a tradução do GT, além de soar estranha, transmite a ideia de que a figura quase evitou uma colisão, mas não conseguiu.
Na tradução do DL, por sua vez, a preposição ``contra'' recupera a informação de que a figura foi de encontro a 
um obstáculo para se chocar contra ele. Porém, em sua totalidade, a tradução soa estranha por evocar uma ideia de sequência entre se chocar contra o muro e, então, levar o carro de volta para os boxes. 
Uma tradução mais apropriada para a sentença original, em nossa avaliação, seria ``Ele conseguiu evitar por muito pouco que o carro se arrebentasse contra o muro, conduzindo-o em seguida com cuidado para os boxes.''. Essa tradução capta a 
informação de que a figura, ``ele'', evitou, por pouco, um choque violento e que conseguiu conduzir o carro de volta aos boxes.


As formalizações apresentadas resumem as principais dificuldades dos tradutores automáticos em capturar aspectos qualitativos da espacialidade.
A Tabela~\ref{table:GT_DL_Errors}, por sua vez, resume a avaliação de cada categoria encontrada em nossa análise: 
tradução (C)orreta, tradução equivocada de (S)ignificado e erro de (P)rojeção na tradução.

\begin{table}[ht]
  \centering
  \footnotesize
  \begin{tabular}{|l|c|c|c|}
  \hline
  & \textbf{Correta} & \textbf{Significado} & \textbf{Projeção} \\
  \hline\hline
  GT & $106~(73.1\%)$ & $19~(13.1\%)$ &  $20~(13.8\%)$\\ \hline
  DL & $118~(81.4\%)$ & $12~(8.3\%)$ &$15~(10.3\%)$ \\ \hline
  Total & $\textbf{224~(77.2\%)}$ & $\textbf{31~(10.6\%)}$ & $\textbf{35~(12.0\%)}$ \\ \hline
  \end{tabular}
  \caption{Categorização da performance do GT e do DL.}
  \label{table:GT_DL_Errors}
  \end{table}

A tabela~\ref{table:GT_DL_Errors} mostra que o DL superou o GT na geração de traduções corretas. O DL traduziu
corretamente $118$~sentenças~($81,4$\%), enquanto o GT traduziu $106$~($73,1$\%). 
O DL também apresentou menos erros de significado~($8,3$\%) e erros de projeção~($10,3$\%) em comparação com 
o GT, que teve~$19$~($13,1$\%) e~$20$~($13,8$\%), respectivamente. Os erros de significado referem-se a situações em que o tradutor automático gera uma frase gramaticalmente correta 
que não transmite o significado original. Por exemplo, ao traduzir \textit{across}, independentemente de seu significado, o GT escolheu predominantemente a tradução ``do outro lado''.
Os erros de projeção, por sua vez, são influenciados por padrões distintos de lexicalização de informações 
espaciais em EN e PT, conforme descrito em~\citet{talmy1985lexicalization, talmy2000toward}. 



\section{Conclusão}



Neste trabalho, analisamos $145$ sentenças que descrevem relaçõs espaciais coletadas dos corpora CAM e COCA e 
traduzidas do EN para o PT pelo Google Translate e pelo DeepL. Utilizando métodos QSR, formalizamos as 
informações espaciais para destacar as diferenças nas relações qualitativas entre as frases de origem e de destino. 
Também analisamos as traduções de todas as sentenças e verificamos que os erros de sentido relacionados à polissemia e os erros de projeção sintática desafiam a tradução automática.

Para tornar nossos resultados mais consistentes, seria interessante (i)~formalizar mais exemplos; 
(ii)~testar computacionalmente as formalizações; e (iii)~analisar traduções automáticas para outros idiomas de destino. 
Uma limitação óbvia dessas sugestões é que elas consomem muito tempo, pois todas as formalizações devem ser feitas manualmente. 
Além disso, reconhecemos a dificuldade de desenvolver métodos automáticos para representar logicamente a linguagem espacial e 
incorporar camadas formais aos modelos de tradução automática neural. De modo geral, esperamos que este trabalho destaque as questões práticas e teóricas da TAN.


\section{Agradecimentos}



\bibliography{references}
\end{document}
