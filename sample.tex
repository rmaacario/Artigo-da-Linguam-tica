\documentclass[a4paper, twocolumn, 11pt, twoside]{article}

\usepackage[brazil]{babel}
\usepackage{linguamatica}
\usepackage{expex}

\usepackage{linguex}
\usepackage[utf8]{inputenc} % To handle special characters
\usepackage{amssymb} % For symbols like ? in glosses


\usepackage[T1]{fontenc}
\usepackage{xcolor}   % For color
\usepackage{graphicx} % For handling images (optional)
\usepackage{fancyhdr} % For custom headers/footers (optional)

\definecolor{ForestGreen}{RGB}{34,139,34}
\definecolor{Maroon}{RGB}{128,0,0}

\definecolor{lightgray}{RGB}{224, 224, 224}
\definecolor{mblue}{RGB}{0, 127, 204}
\definecolor{lightyellow}{RGB}{255, 247, 209}
\definecolor{lightblue}{RGB}{204, 230, 255} % 

%% Please use UTF8 in your document

%% Set the bibliography style.
%% Styles provided for Spanish (Castilian), Catalan and Portuguese.
%% Other languages will be created on a required basis.

\bibliographystyle{sp_por}   % This for Portuguese
% \bibliographystyle{sp_esp} % This for Castilian
% \bibliographystyle{sp_cat} % This for Catalan


%% You can leave this unchanged. It will be updated by the editors when your paper gets published.
\submitted{15 de \OCT{} 2024}
\accepted{3 de \DEC{} 2024}


%% Add your title in the main language used in the article
\title{LLMs vs. NMTs: Desafios na Tradução da Linguagem Espacial EN-PT-br em Legendas de TED Talks}

%% Currenty the title in English is also mandatory
\titleEN{LLMs vs. NMTs: Challenges in Translating Spatial Language EN-PT-br in TED Talk Subtitles}


%% Add authors here. Three lines for each author.
%% First line with author name, second with the affiliation and third with e-mail
%% In cases where a second line is needed for the affiliation, use the command \nl to separate them. 
\author{
  Rafael Fernandes
  \instituto{Universidade de São Paulo}
  \email{rafael.macario@usp.br} 
  \and 
%%  Rodrigo Souza
%%  \instituto{Universidade de São Paulo}
%%  \email{rodrigo.aparecido.souza@usp.br}
%%  \and
  Marcos Lopes
  \instituto{Universidade de São Paulo}
  \email{marcoslopes@usp.br}
  \and
%%  Paulo Santos
%%  \instituto{Flinders University}
%%  \email{paulo.santos@flinder.edu.au}
%%%%  \and
%%  Thomas Finbow
%%  \instituto{Universidade de São Paulo}
%%  \email{thomas.finbow@usp.br}
}


\begin{document}
\maketitle

%% Add the abstract in the main language for the article. If possible, doesn't add any citations here.
\begin{resumo}
Este estudo explora os desafios de traduzir a semântica espacial em legendas do inglês para o português brasileiro (EN-PT-br). Ao comparar o desempenho de Modelos de Linguagem de Código Aberto (LLMs) com sistemas de Tradução Automática Neural (NMT), analisamos como esses modelos lidam com preposições espaciais e outros elementos semanticamente ricos. Através de uma combinação de métricas de avaliação automatizadas e análise manual de erros, o estudo identifica as forças e limitações de cada tipo de modelo, oferecendo insights sobre o futuro da tecnologia de tradução.
\end{resumo}

%% Add keywords in the article main language, in lowercase
\palavraschave{exemplo, pln, linguamática}


%% Add the abstract in English
\begin{abstract}
This study explores the challenges of translating spatial semantics in subtitles from English to Brazilian Portuguese (EN-PT-br). By comparing the performance of open-source Large Language Models (LLMs) with Neural Machine Translation (NMT) systems, we analyze how effectively these models handle spatial prepositions and other semantically-rich elements. Through a combination of automated evaluation metrics and manual error analysis, the study identifies the strengths and limitations of each model type, offering insights into the future of translation technology.
\end{abstract}

%% add the keywords in English
\keywords{Open-source LLMs, Neural Machine Translation, Spatial Semantics, Polysemy, Language Typology}



\section{Introdução}

A tradução da semântica espacial ainda enfrenta desafios significativos no campo da tradução automática, especialmente ao lidar com preposições que expressam relações espaciais complexas. Preposições como ACROSS, INTO, ONTO e THROUGH, que possuem traduções altamente dependentes do contexto, são difíceis de traduzir com precisão por máquinas entre línguas tipologiamente distintas. Esse desafio se torna especialmente evidente ao traduzir do inglês (EN), uma língua \texttt{S-framed}, para o português brasileiro (PT-br), uma língua \texttt{V-framed}, onde a ênfase nas relações de movimento e espaço é fundamentalmente distinta \citep{talmy2000toward,talmy2000towardb, Slobin-2004, slobin2005relating, Slobin2006WhatMM}. A tradução correta dessas preposições é crucial, pois erros podem resultar em traduções imprecisas, especialmente em contextos de legendas, onde a concisão e a clareza são essenciais. O Exemplo~\ref{ex:ex-1} ilustra esse desafio.

\ex \texttt{(inner\_id: 27316)} \hfill \texttt{Across(ii)} \\[0.3ex]
\noindent\rule{\linewidth}{0.9pt}
When I \colorbox{lightblue}{\textcolor{blue}{walked}} \colorbox{lightgray}{\textcolor{blue}{\emph{across}}} \colorbox{lightgray}{\textcolor{blue}{Afghanistan}}, I \textcolor{blue}{stayed} with people like this. (SRC) \label{ex:ex-1} \\[-0.3ex]
\noindent\rule{\linewidth}{0.3pt} \\ 
Ele \textit{atravessou} a multidão com dificuldade até chegar à frente.  (REF) \\ 
\gll 3SG.M \textit{crossed} the crowd with difficulty until reaching the front \\
\medskip
\a. ?~Ele lutou no meio da multidão até chegar à frente. (GT) \\ 
\gll 3SG.M fought in-the middle of-the crowd until reach to-the front \\
\medskip
\b. ?~Ele se debateu entre a multidão até chegar à frente. (DL) \\ 
\gll 3SG.M REFL struggled amongst the crowd until reach to-the front \\
\medskip
\c. Ele lutou contra a multidão até chegar à frente. (GM) \\
\gll 3SG.M He \textit{crossed} the crowd with difficulty until reaching the front \\
\d. Ele lutou para passar pela multidão até chegar à frente.
%\par
\normalsize


Os avanços recentes nos tradutores automáticos neurais (NMTs) e o surgimento dos grandes modelos de linguagem (LLMs) têm oferecido novas ferramentas para enfrentar esses desafios. Embora os NMTs tradicionais tenham progredido significativamente na melhoria da qualidade da tradução, ainda enfrentam dificuldades com as sutilezas da linguagem espacial, resultando, frequentemente, em traduções excessivamente literais ou idiomaticamente imprecisas. Por outro lado, os LLMs, com seus vastos conjuntos de dados de treinamento e arquiteturas sofisticadas, têm o potencial de produzir traduções mais adequadas ao contexto. No entanto, o desempenho desses modelos na tradução do domínio espacial ainda é pouco explorado, especialmente quando comparado aos sistemas NMT tradicionais.

Este artigo investiga como diferentes modelos lidam com a tradução de preposições espaciais do inglês para o português brasileiro (EN-PT-BR). Focando em legendas de TED Talks, esta pesquisa busca avaliar a eficácia dos LLMs em preservar a integridade semântica das expressões espaciais, em comparação com os sistemas NMT tradicionais. Os resultados visam contribuir para o desenvolvimento contínuo das tecnologias de tradução automática, identificando áreas em que esses modelos se destacam e onde ainda há necessidade de melhorias.

\subsection{Desafios na Tradução da Espacialidade}

A formatação ao longo do documento é a normal em documentos \LaTeX, sem grandes alterações. No entanto,
algumas sugestões:
\begin{itemize}
\item Para dar \emph{ênfase} use sempre que possível o comando \verb.\emph.;
\item Para citar poderá usar o comando \verb.\citep. que cria referências entre
  parêntesis~\citep{latex}. Para citar um \citet{autor}, use o comando \verb.\citet.;
\item Citações seguidas devem reaproveitar o comando de citação. Caso necessite de indicar a página
  a que se refere a citação, use \citep[p.~40]{latex}.
\item Ao criar entradas bibliográficas assegure-se da correção do seu
  conteúdo. Não abrevie nomes de autores. Não coloque os nomes dos
  editores de livros de atas. Não se esqueça dos números das páginas
  do documento.
\item Sempre que usar endereços web e outros tipos de URI, coloque-os com o comando \verb.\url. e,
  sempre que possível, em nota de fim de página.\footnote{Assim. \url{http://www.linguamatica.com}}
\item Nas notas de fim de página que sejam anexadas a palavras seguidas de pontuação, devem ser colocadas
  após a pontuação, como exemplificado no item anterior.
\item Tenha em atenção a diferença entre -, -- e ---. O primeiro será usado entre palavras, como em
  curto-circuito, o segundo em intervalos, como 10--20 ou PT--EN e o terceiro --- este --- para
  introduzir pequenos comentários.
\item As figuras devem ser legendadas e a legenda deve terminar com um sinal de pontuação.
\item As referências a figuras, tabelas ou secções devem ser criadas usando as ferramentas do \LaTeX{}.
\item Sempre que possivel garanta a qualidade das imagens importadas, usando PDF ou PNG.
\item Ao criar tabelas (Tabela~\ref{tab:1}) tente diminuir a quantidade de traços usada. Grande parte das
  tabelas são legíveis apenas com um par de linhas como demonstrado.
\end{itemize}

\begin{table}[htb]
  \centering
  \begin{tabular}{lrr}
    \toprule
    & \textbf{Homens} & \textbf{Mulheres} \\
    \midrule
    Crianças & 10~032 & 32~341 \\
    Adultos & 23~431 & 9~443 \\
    \bottomrule
  \end{tabular}
  \caption{Exemplo de tabela com poucos traços.}
  \label{tab:1}
\end{table}



\section*{Agradecimientos}

Os agradecimentos devem ser colocados sempre numa secção final, sem número, tal como
neste exemplo. Sempre que o autor assim o entender, deverá agradecer aos revisores.


\bibliography{references.bib}


\end{document}
